% SPDX-FileCopyrightText: Copyright (C) Nile Jocson <novoseiversia@gmail.com>
% SPDX-License-Identifier: MPL-2.0

\documentclass{article}

% SPDX-FileCopyrightText: Copyright (C) Nile Jocson <novoseiversia@gmail.com>
% SPDX-License-Identifier: MPL-2.0

\usepackage{adjustbox}
\usepackage{amsmath}
\usepackage{amssymb}
\usepackage[a4paper, margin=1in]{geometry}
\usepackage{pgfplots}
\usepackage{physics}
\usepackage{tabularray}
\usepackage{tkz-tab}
\usepackage{xpatch}



\renewcommand{\arraystretch}{1.75}
\renewcommand{\thesubsubsection}{\thesubsection.\alph{subsubsection}}



\DefTblrTemplate{caption}{default}{}
\DefTblrTemplate{capcont}{default}{}

\xpatchcmd{\tkzTabLine}{$0$}{$\bullet$}{}{}
\tikzset{t style/.style={style=solid}}



\newcommand*{\cjboilerplate}[2]{
	\author{Nile Jocson \textless{}novoseiversia@gmail.com\textgreater{}}
	\title{Exercise Solutions for #1\\{\large #2}}
	\date{\today}

	\maketitle{}
	\null\vfill\noindent
	Copyright \copyright{} Nile Jocson \textless{}novoseiversia@gmail.com\textgreater{} \\
	Licensed under MPL-2.0. See LICENSE file.
		\pagebreak
}



\newenvironment{cjsection}[1]
{
	\section{#1}
}
{
	\pagebreak
}

\newcommand*{\cjitem}[1]{\subsection{#1}}
\newcommand*{\cjsubitem}[1]{\subsubsection{#1}}



\newcommand*{\cjsolsect}[1]{\hline --- #1: \\}
\newcommand*{\cjneeded}[2]{\hline --- Needed: \\ \(\square\) #1 \(= \mathord{?}\) & #2 \\}
\newcommand*{\cjgiven}[2]{\(\square\) #1 & #2 \\}

\newcommand*{\cjwhy}[2]{\hline \(\Rightarrow\) #1 & #2 \\}
\newcommand*{\cjsubwhy}[2]{\(\Rightarrow\) #1 & #2 \\}
\newcommand*{\cjcontinue}[1]{\(\Rightarrow\) #1 & \\}
\newcommand*{\cjfa}[1]{\hline \(\Rightarrow\) #1 & Final answer. \\}
\newcommand*{\cjfastep}[2]{\hline \(\Rightarrow\) #1 & Final answer. #2 \\}
\newcommand{\cjsign}[1]{\hline & Create a table of signs. \\ \begin{adjustbox}{width=0.49\textwidth}\begin{tikzpicture}#1\end{tikzpicture}\end{adjustbox} \\ \\}
\newcommand{\cjgraph}[1]{\hline & Graph the equation. \\ \begin{adjustbox}{width=0.49\textwidth}\begin{tikzpicture}\begin{axis}[axis lines=middle, axis equal, grid=both]#1\end{axis}\end{tikzpicture}\end{adjustbox} \\ \\}

\newcommand*{\cjqed}{\(\blacksquare\)}

\NewDocumentEnvironment{cjsolution}{+b}
{
	\begin{longtblr}
	[
		expand = \cjwhy\cjsubwhy\cjcontinue\cjfa\cjfastep\cjsign\cjgraph\cjgiven\cjsolsect\cjneeded
	]
	{
		colspec = {|lX[r]|},
		width = \textwidth
	}
		#1
		& \cjqed{} \\
		\hline
	\end{longtblr}
}{}

\newcommand*{\cjdiv}{\divisionsymbol{}}
\newcommand*{\cjexp}[1]{\times 10^{#1}}
\newcommand*{\cjunit}[1]{\text{ #1}}
\newcommand*{\cjceil}[1]{\lceil#1\rceil}
\newcommand*{\cjlog}[2]{\text{log}_{#1} #2}




\begin{document}
	\cjboilerplate{EEE 113}{Communications - Basics of Communication Systems}

	\begin{cjsection}{}
		\cjitem{Suppose we sample a signal at frequency \(F_s\). If we collect 9.1 kilo-samples in
		4.7 seconds, what is \(F_s\) in Hz (samples/second)? Please provide your answer in at least
		5 significant figures.}
			\begin{cjsolution}
				\cjneeded{\(F_s\)}{Sampling frequency.}
				\cjsolsect{Given}
				\cjgiven{\(S = 9.1\cjunit{kilo-samples}\)}{Amount of samples.}
				\cjgiven{\(t = 4.7\cjunit{seconds}\)}{Amount of time.}

				\cjwhy{\(F_s = \frac{9.1\cjunit{kilo-samples}}{4.7\cjunit{seconds}}\)}{\(F_s = \frac{S}{t}\)}
				\cjfa{\(F_s = 1936.1\cjunit{Hz}\)}
			\end{cjsolution}

		\cjitem{Compact discs record two channels (left and right) of music at a sampling frequency
		of \(F_s = 44.1\text{ kHz}\) for each channel. If each sample is encoded with 16 bits, and
		one byte is 8 bits, how many bytes are required to store 51.2 seconds of music?}
			\begin{cjsolution}
				\cjneeded{\(B_T\)}{Total amount of bytes needed.}
				\cjsolsect{Given}
				\cjgiven{\(c = 2\cjunit{channels}\)}{Amount of channels.}
				\cjgiven{\(F_s = 44.1\cjunit{kHz/channel}\)}{Sampling frequency.}
				\cjgiven{\(b_s = 16\cjunit{bits/sample}\)}{Amount of bits in a sample.}
				\cjgiven{\(b_B = 8\cjunit{bits/byte}\)}{Amount of bits in a byte.}
				\cjgiven{\(t = 51.2\cjunit{seconds}\)}{Amount of time.}
				\cjsolsect{Let}
				\cjgiven{\(F_T\)}{Total sampling frequency.}
				\cjgiven{\(S_T\)}{Total amount of samples.}
				\cjgiven{\(b_T\)}{Total amount of bits.}

				\cjwhy{\(F_T = 44.1\cjunit{kHz/channel} \cdot 2\cjunit{channels}\)}{Note that we need two channels.}
				\cjcontinue{\(F_T = 88.2\cjunit{kHz}\)}
				\cjwhy{\(S_T = 88.2\text{ kHz} \cdot 51.2\cjunit{seconds}\)}{\(S_T = F_Tt\)}
				\cjcontinue{\(S_T = 4.51584\cjexp{6}\cjunit{samples}\)}
				\cjwhy{\(b_T = 4.51584\cjexp{6}\cjunit{samples} \cdot 16\cjunit{bits/sample}\)}{\(b_T = S_Tb_s\)}
				\cjcontinue{\(b_T = 72.25344\cjexp{6}\cjunit{bits}\)}
				\cjwhy{\(B_T = \frac{72.25344\cjexp{6}\cjunit{bits}}{8\cjunit{bits/byte}}\)}{\(B_T = \frac{b_T}{b_B}\)}
				\cjfa{\(B_T = 9.03168\cjexp{6}\cjunit{bytes}\)}
			\end{cjsolution}

		\cjitem{Consider a system that uses 8-bit ASCII codes to encode letters. How long (in
		microseconds) will it take to transmit the bit sequence encoding ``Hello-World!'' if we
		use a bit time of 4 samples per bit, and transmit samples at a rate of 1 MHz?}
			\begin{cjsolution}
				\cjneeded{\(t\)}{Amount of time needed.}
				\cjsolsect{Given}
				\cjgiven{\(b_c = 8\cjunit{bits/character}\)}{Amount of bits per character.}
				\cjgiven{\(c = 12\cjunit{characters}\)}{Amount of characters in the string.}
				\cjgiven{\(b_t = 4\cjunit{samples/bit}\)}{Bit time.}
				\cjgiven{\(F_{Tx} = 1\cjunit{MHz}\)}{Transmission frequency.}
				\cjsolsect{Let}
				\cjgiven{\(b_T\)}{Total amount of bits.}
				\cjgiven{\(S_T\)}{Total amount of samples.}

				\cjwhy{\(b_T = 12\cjunit{characters} \cdot 8\cjunit{bits/character}\)}{\(b_T = cb_c\)}
				\cjcontinue{\(b = 96\cjunit{bits}\)}
				\cjwhy{\(S_T = 96\cjunit{bits} \cdot 4\cjunit{samples/bit}\)}{\(S_T = b_Tb_t\)}
				\cjcontinue{\(S_T = 384\cjunit{samples}\)}
				\cjwhy{\(t = \frac{384\cjunit{samples}}{1\cjunit{MHz}}\)}{\(t = \frac{S_T}{F_{Tx}}\)}
				\cjfa{\(t = 384\cjunit{microseconds}\)}
			\end{cjsolution}

		\cjitem{The process of converting a discrete-time continuous-valued signal into a digital
		signal by expressing each sample as a finite number of digits is called (a); (b) is the
		difference between sampled analog signal and digitized signal; (c) refers to the difference
		between the minimum and maximum value of the discrete-time signal; (d) is the assignment of
		a unique binary number to each quantization level.}
			\begin{cjsolution}
				\cjfa{a: quantization}
				\cjcontinue{b: quantization error}
				\cjcontinue{c: dynamic range}
				\cjcontinue{d: coding}
			\end{cjsolution}

		\cjitem{An analog waveform is described as \[x(t) = 2u(t) - 5u(t - 1) + 3u(t - 2) + 3u(t - 3) - 2u(t - 4) - u(t - 5)\]
		where \(u(t)\) is the continuous-time unit step function. If the quantization level is
		0.02, determine the minimum number of bits needed to represent the full range of the analog signal \(x(t)\).}
			\begin{cjsolution}
				\cjneeded{\(b_m\)}{Minimum amount of bits needed.}
				\cjsolsect{Given}
				\cjgiven{\(Q = 0.02\)}{Quantization step size.}
				\cjgiven{\(M = 3\)}{Maximum value of the waveform.}
				\cjgiven{\(m = -5\)}{Minimum value of the waveform.}
				\cjsolsect{Let}
				\cjgiven{\(R\)}{Dynamic range.}
				\cjgiven{\(L\)}{Amount of quantization levels.}

				\cjwhy{\(R = 3 - (-5)\)}{\(R = M - m\)}
				\cjcontinue{\(R = 8\)}
				\cjwhy{\(L = \frac{8}{0.02}\)}{\(L = \frac{R}{Q}\)}
				\cjcontinue{\(L = 400\cjunit{levels}\)}
				\cjwhy{\(b_m = \cjceil{\cjlog{2}{400\cjunit{levels}}}\)}{\(B_m = \cjceil{\cjlog{2}{L}}\)}
				\cjcontinue{\(b_m = \lceil8.64385618977\rceil\)}
				\cjfa{\(b_m = 9\cjunit{bits}\)}
			\end{cjsolution}
	\end{cjsection}
\end{document}
